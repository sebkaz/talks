\documentclass[12pt]{article}
\usepackage{calc}
\usepackage{color}
\usepackage{amsfonts}
\usepackage{latexsym}
\usepackage{placeins}
\ifx\pdftexversion\undefined
  \usepackage[dvips]{graphicx}
\else
  \usepackage[pdftex]{graphicx}
\fi
\usepackage{amssymb}
\usepackage{authblk}
\usepackage{amsmath}
\usepackage[utf8]{inputenc}
\usepackage[OT4]{fontenc}

\addtolength{\voffset}{-3.5cm} \addtolength{\textheight}{4cm}

\renewcommand\Authfont{\scshape\small}
\renewcommand\Affilfont{\itshape\small}
\setlength{\affilsep}{1em}

\newcommand{\smalllineskip}{\baselineskip=15pt}
\newcommand{\keywords}[1]{{\footnotesize\hspace{0.68cm}{\textit{Keywords}: }#1\par
  \vskip.7\baselineskip}}
\renewenvironment{abstract}[0]{\small\rm
        \begin{center}ABSTRACT
        \\ \vspace{8pt}
        \begin{minipage}{5.2in}\smalllineskip
        \hspace{1pc}}{\end{minipage}\end{center}\vspace{-1pt}}
\newcommand{\emailaddress}[1]{\newline{\sf#1}}

\let\LaTeXtitle\title
\renewcommand{\title}[1]{\LaTeXtitle{\large\textsf{\textbf{#1}}}}

%%%TITLE
\title{Chord diagrams category and its limit}
\date{}

%%AFFILIATIONS
\author[1]{Sebastian Zając}
\affil[1]{Faculty of Mathematics and Natural Studies Cardinal Stefan Wyszynski University in Warsaw Dewajtis 5, Warszawa, Poland \emailaddress{s.zajac@uksw.edu.pl}}

%%DOCUMENT
\begin{document}
\maketitle

%%PLEASE PUT YOUR ABSTRACT HERE
\begin{abstract}{\it Chord diagrams} may be used to represent the inter-reletionships between some objects. They occur in many branches of matematics and physics like: geometry, topology, random matrix models, moduli space of Riemann surfaces. In physics they may be used to represent interactions between particles (Feynman diagrams).  Chord diagrams are also very important in a complicated problem of RNA and protein structure prediction in molecular biology. They represent so called secondary structure interactions and can be used to describe properties of this structures by a topological characteristic called genus.  Category theory is a natural framework for a description of chord diagrams. Because of amalgamation properties of this category it is possible to find so called Fraisse limit and apply Fraisse theory to them. I will also present some questions about category theory and they role in mathematical physics, logic and philosophy.


\end{abstract}
%%THE END OF ABSTRACT

\end{document}
